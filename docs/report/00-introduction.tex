\phantomsection\section*{ВВЕДЕНИЕ}\addcontentsline{toc}{section}{ВВЕДЕНИЕ}

Выделение страниц для разных типов данных в ядре операционной системы GNU/Linux привело к явлению, известному как скрытая фрагментация.
Это явление возникает из-за того, что выделенные, но не полностью используемые страницы могут оставаться частично занятыми, что приводит к неэффективному использованию физической памяти.
Для решения этой проблемы в ядре Linux был разработан механизм SLAB-кэша, который позволяет эффективно управлять памятью, минимизируя внутреннюю фрагментацию и ускоряя процесс выделения и освобождения объектов.

Одним из важных аспектов управления памятью в операционных системах является мониторинг выделения физической памяти и работы механизмов управления памятью, включая SLAB-кэш.
Понимание того, как ядро использует память, позволяет оптимизировать работу системы, находить утечки памяти и повышать эффективность использования ресурсов.

Целью данной работы является разработка загружаемого модуля ядра для мониторинга выделения памяти в адресном пространстве ядра, предоставляющий информацию о запросах выделения физической памяти и выделения памяти в SLAB-кэше.

%Современные веб-сайты, как правило, состоят из веб-сервера, отвечающего за обработку входящих HTTP запросов, отдачу клиенту статических файлов и проксирование запроса до бэкенд-серверов.
%Благодаря такому подходу бэкенд-сервера занимаются только обработкой некой бизнес-логики и походами в базы данных, и не тратят свои ресурсы на отдачу файлов, для генерации которых не нужна какая-то особая обработка.~\cite{aboba}
%
%Целью данной работы является разработка сервера для отдачи статического содержимого с диска.
%
%Для достижения поставленной цели необходимо решить следующие задачи:
%\begin{itemize}
%    \item провести анализ предметной области;
%    \item разработать алгоритмы работы сервера;
%    \item выбрать средства реализации и реализовать разработанные алгоритмы;
%    \item провести исследование зависимости общего времени обслуживания запросов от их количества и сравнить полученные результаты с сервером nginx.
%\end{itemize}
