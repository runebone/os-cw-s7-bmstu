\section{Исследовательский раздел}

%В данном разделе будут приведены описание исследования, технические характеристики устройства, на котором оно проводилось, а также его результаты.
%
%\subsection{Описание исследования}
%
%Запросы к серверам осуществлялись от двадцати различных клиентов с помощью инструмента Apache Benchmark~\cite{ab}.
%Запрашивался текстовый файл размером 25МБ.
%
%Команда, с помощью которой осуществлялись запросы к серверам:
%
%\begin{lstlisting}[numbers=none]
%ab -n N -c 20 http://localhost:8080/large.txt
%\end{lstlisting}
%
%где $N \in \{250, 500, 750, 1000, 1250, 1500, 1750, 2000\}$ --- количество запросов, 20 --- количество клиентов, создающих запросы.
%
%\subsection{Технические характеристики}
%
%Технические характеристики устройства, на котором выполнялись замеры времени, представлены ниже.
%\begin{enumerate}
%    \item Процессор: \texttt{AMD Ryzen 7 4700U} 2.0 ГГц~\cite{amd}, 8 физических ядер, 8 потоков;
%    \item Оперативная память: 8 ГБ, \texttt{DDR4}, 3200 МГц;
%    \item Операционная система: \texttt{Arch Linux}~\cite{arch};
%    \item Версия ядра: \texttt{6.12.10}.
%\end{enumerate}
%
%При выполнении замеров времени ноутбук был подключен к сети электропитания, были открыты два терминала Alacritty~\cite{alacritty}, в которых были запущены Apache Benchmark и Neovim.
%На фоне работал Docker-контейнер~\cite{docker} с nginx~\cite{nginx}.
%
%\subsection{Результаты исследования}
%
%Результаты проведенного исследования отображены на рисунке \ref{fig:plot} и в таблице \ref{tab:table}.
%
%\begin{figure}[H]
%	\centering
%	\includegraphics[scale=0.9]{img/plot.pdf}
%	\caption{Зависимость общего времени обработки запросов от количества запросов}
%	\label{fig:plot}
%\end{figure}
%
%\begin{table}[H]
%    \centering
%    \caption{Зависимость общего времени обработки запросов от количества запросов}
%    \label{tab:table}
%    \begin{tabular}{|c|c|c|}
%        \hline
%        \textbf{Количество запросов} & \textbf{Разработанный сервер (с)} & \textbf{nginx (с)} \\ \hline
%        250                          & 3.141                 & 3.774              \\ \hline
%        500                          & 6.282                 & 7.386              \\ \hline
%        750                          & 9.376                 & 11.610             \\ \hline
%        1000                         & 12.510                & 15.239             \\ \hline
%        1250                         & 15.685                & 18.599             \\ \hline
%        1500                         & 18.828                & 22.134             \\ \hline
%        1750                         & 21.866                & 26.472             \\ \hline
%        2000                         & 25.011                & 29.502             \\ \hline
%    \end{tabular}
%\end{table}
%
%Как можно заметить по графику \ref{fig:plot} и таблице \ref{tab:table}, общее время обработки запросов разработанным сервером в среднем меньше общего времени обработки запросов nginx приблизительно в ${5}/{6}$ раза.
%
%\subsection*{Вывод}
%
%В результате проведенного исследования было установлено, что общее время обработки запросов разработанным сервером в среднем меньше общего времени обработки запросов nginx приблизительно в ${5}/{6}$ раза, а также, по мере увеличения количества запросов, можно наблюдать линейный рост общего времени обработки запросов для обоих серверов.
