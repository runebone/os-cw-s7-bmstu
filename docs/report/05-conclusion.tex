\phantomsection\section*{ЗАКЛЮЧЕНИЕ}\addcontentsline{toc}{section}{ЗАКЛЮЧЕНИЕ}

Цель данной работы, а именно разработка загружаемого модуля ядра для мониторинга выделения памяти в адресном пространстве ядра, предоставляющего информацию о запросах выделения физической памяти и выделения памяти в SLAB-кэше, была достигнута.

Для решения поставленной цели были решены следующие задачи:
\begin{enumerate}
    \item проведен анализ способов выделения памяти в ядре Linux;
    \item проведен анализ способов мониторинга выделения памяти;
    \item выбраны функции и механизмы для реализации загружаемого модуля ядра, позволяющего осуществлять мониторинг выделения памяти в адресном пространстве ядра;
    \item разработаны алгоритмы для решения поставленной задачи;
    \item реализован загружаемый модуль ядра, решающий поставленную задачу;
    \item разработанное программное обеспечение было протестировано.
\end{enumerate}

В результате проведенного исследования количества объектов в кэшах kmalloc-256, kmalloc-128 и kmalloc-64 до, во время и после запуска программы для генерации потоков, было установлено, что во всех трех кэшах наблюдается значительное увеличение количества объектов в период выполнения программы.
Наибольший рост зафиксирован в kmalloc-64 (на 144.3\%), что указывает на активное выделение небольших блоков памяти.
