\documentclass[gray]{beamer}
\usepackage[utf8]{inputenc}
\usepackage[T2A]{fontenc}
\usepackage[english,russian]{babel}
\usepackage{newtxtext,newtxmath}
\usepackage{graphicx} % Пакет для работы с изображениями
\usepackage{adjustbox} % Hyperlinks
\usepackage{courier}
\usepackage{blindtext}

\usepackage{float}
\usepackage{listings}

\definecolor{c}{HTML}{000050}

\usetheme{default}

\setbeamertemplate{footline}[frame number]
\setbeamertemplate{navigation symbols}{}

\begin{document}

\begingroup
\setbeamertemplate{footline}{}
\begin{frame}
%\begin{center}
    % \hfill
    \begin{minipage}{0.1\textwidth}
        \includegraphics[width=1.7cm]{img/bmstu.pdf}
    \end{minipage}
    % \hspace{1cm}
    \hfill
    \begin{minipage}{0.8\textwidth}\centering\bfseries
        {
            \linespread{1}\selectfont\tiny
            \vspace{0.1cm}
            {Министерство~науки~и~высшего~образования~Российской~Федерации}

            {Федеральное~государственное~бюджетное~образовательное~учреждение высшего~образования}

            {
                <<Московский~государственный~технический~университет

                имени~Н.Э.~Баумана

                (национальный~исследовательский~университет)>>
            }

            {(МГТУ им. Н.Э.~Баумана)}
            \vspace{0.1cm}
        }
    \end{minipage}

    \vspace{0.2cm}
    %\rule{\linewidth}{2.8pt}
    \rule[3ex]{\linewidth}{1pt}

    \vfill

    \begin{center}
    \Large Разработка сервера для отдачи статического содержимого с диска
    \end{center}

    \vfill

    \begin{minipage}[t]{0.45\textwidth}
        \small
        \raggedright
        Выполнил:

        студент 4 курса

        группы ИУ7-74Б

        \makebox[0pt][l]{Рунов Константин Алексеевич}
    \end{minipage}
    \hfill
    \begin{minipage}[t]{0.45\textwidth}
        \small
        \raggedleft
        Руководитель:

        \makebox[0pt][r]{Клочков Максим Николаевич}
    \end{minipage}

    \vfill

    \begin{center}
        Москва, \the\year\ г.
    \end{center}
\end{frame}
\endgroup

\begin{frame}
    \frametitle{Цель и задачи}
    Целью данной работы является разработка сервера для отдачи статического содержимого с диска.

    \vfill

    Задачи:
    \begin{itemize}
        \item провести анализ предметной области;
        \item разработать алгоритмы работы сервера;
        \item выбрать средства реализации и реализовать разработанные алгоритмы;
        \item провести исследование зависимости общего времени обслуживания запросов от их количества и сравнить полученные результаты с сервером nginx.
    \end{itemize}
\end{frame}

\begin{frame}
    \frametitle{Анализ предметной области. Статический веб-сервер}
    \centering
    \includegraphics[height=0.9\textheight]{diag/webserver.pdf}
\end{frame}

\begin{frame}
    \frametitle{Анализ предметной области. Мультиплексирование}
    \centering
    \includegraphics[width=\textwidth]{img/select.png}
\end{frame}

\begin{frame}
    \frametitle{Алгоритм работы сервера --- main}
    \centering
    \includegraphics[height=0.8\textheight]{diag/main.pdf}
\end{frame}

\begin{frame}
    \frametitle{Алгоритм работы сервера --- spawn\_workers}
    \centering
    \includegraphics[height=0.8\textheight]{diag/spawn_workers.pdf}
\end{frame}

\begin{frame}
    \frametitle{Алгоритм работы сервера --- handle\_client}
    \centering
    \includegraphics[height=0.8\textheight]{diag/handle_client.pdf}
\end{frame}

\begin{frame}
    \frametitle{Средства реализации}
    %\centering
    Язык программирования: C

    Среда разработки: Neovim
\end{frame}

\begin{frame}
    \frametitle{Результаты нагрузочного тестирования}
    \centering
    \includegraphics[height=0.8\textheight]{img/plot.pdf}
\end{frame}

\begin{frame}
    \frametitle{Сравнительная таблица реализованного веб-сервера и веб-сервера nginx}
    \centering

    \begin{table}[H]
        \centering
        \caption{Зависимость общего времени обработки запросов от количества запросов}
        \label{tab:table}
        \begin{tabular}{|c|c|c|}
            \hline
            \textbf{Количество запросов} & \textbf{Разработанный сервер (с)} & \textbf{nginx (с)} \\ \hline
            250                          & 3.141                 & 3.774              \\ \hline
            500                          & 6.282                 & 7.386              \\ \hline
            750                          & 9.376                 & 11.610             \\ \hline
            1000                         & 12.510                & 15.239             \\ \hline
            1250                         & 15.685                & 18.599             \\ \hline
            1500                         & 18.828                & 22.134             \\ \hline
            1750                         & 21.866                & 26.472             \\ \hline
            2000                         & 25.011                & 29.502             \\ \hline
        \end{tabular}
    \end{table}
\end{frame}

\begin{frame}
    \frametitle{Заключение}
Цель данной работы, а именно разработка сервера для отдачи статического содержимого с диска, была достигнута.

Для достижения поставленной цели были решены следующие задачи:
\begin{itemize}
    \item проведен анализ предметной области;
    \item разработан алгоритмы работы сервера;
    \item выбраны средства реализации и реализовать разработанные алгоритмы;
    \item проведено исследование зависимости общего времени обслуживания запросов от их количества и сравнение полученных результатов с сервером nginx.
\end{itemize}
\end{frame}

\begin{frame}
    \frametitle{Заключение}
В результате проведенного исследования было установлено, что общее время обработки запросов разработанным сервером в среднем меньше общего времени обработки запросов nginx приблизительно в ${5}/{6}$ раза, а также, по мере увеличения количества запросов, можно наблюдать линейный рост общего времени обработки запросов для обоих серверов.
\end{frame}

\end{document}
